\documentclass[a4j,11pt,twocolumn]{jsarticle}

%\usepackage{multicol} %2段組み用
\usepackage{amsmath}  %alignなど数式関係で必要になることが多い.
%\usepackage{graphicx} %図の取り込みに利用.
\usepackage[dvipdfmx]{graphicx} %図の取り込みに利用.
\usepackage{url}      %URLの表記に使う\urlコマンドに必要.
\usepackage{txfonts}  %英文をTimes Romanのようなフォントにする.
                      %通常のLaTeXのフォントにしたいときはこれをコメントアウトする.
\usepackage{algorithm,algorithmic}
\usepackage{enumerate}

\pagestyle{plain} %ページ番号のスタイル
\urlstyle{same}   %\urlコマンドのフォント指定."tt","rm","sf","same"(=使用中のフォント)

%%%%%%%%%%%%%%%%%%%%%%%%%%wsz%%%% ↓テキスト幅,マージン,行間の調節
\textwidth     =  160mm %テキスト幅
\oddsidemargin = -7.5mm %左側のマージン
\textheight    =  230mm %テキストの高さ
\topmargin     =  -15mm %上のマージン
\renewcommand{\baselinestretch}{1.1} %行間を調節
%%%%%%%%%%%%%%%%%%%%%%%%%%%%%% ↑テキスト幅,マージン,行間の調節

%%%% ↓Change the style of itemize,enumerate,etc.
%%%% ↓(without space between items)
\makeatletter
\def\@listI{\leftmargin\leftmargini
    \topsep  \z@
    \parsep  \z@
    \itemsep \z@}
\let\@listi\@listI
\@listi
\def\@listii{\leftmargin\leftmarginii
    \labelwidth\leftmarginii\advance\labelwidth-\labelsep
    \topsep  \z@
    \parsep  \z@
    \itemsep \z@}
\def\@listiii{\leftmargin\leftmarginiii
    \labelwidth\leftmarginiii\advance\labelwidth-\labelsep
    \topsep  \z@
    \parsep  \z@
    \itemsep \z@}
\makeatother
%%%% ↑Change the style of itemize,enumerate,etc.

%%%% ↓algotithmic の \REQUIRE と \ENSURE の表記を変更する
\renewcommand{\algorithmicrequire}{\textbf{Input:}}
\renewcommand{\algorithmicensure}{\textbf{Output:}}
%%%% ↑algotithmic の \REQUIRE と \ENSURE の表記を変更する

\newcommand\coleq{\mathrel{\mathop:}=}% \coleqの定義(:=をきれいに出力する)

\begin{document}
\twocolumn[%
\begin{center}
 {\LARGE \textbf{自動車運搬船における貨物積載プランニングの席割問題}}\\
 %\vspace{5mm}
 {\Large 竹田 陽} \\
 2021年4月21日
\end{center}
\begin{quote}
 \textbf{Abstract.}
 Our research considers the situation in which cars are transported to several ports of various countries by a carrier ship for cars (pure car carrier, PCC). Two stages of work called \textit{stowage planning} and \textit{simulation planning} are performed before loading a collection of cars onto a PCC. The stowage planning stage considers assigning cars to one of the areas called \emph{holds}, a few of which compose each level of the ship. The simulation planning stage places the cars assigned to each area in the stowage planning one by one considering the orientation of cars. In our research, as the first step of fully automating the stowage planning and simulation planning, the goal is to output stowage plans efficiently with a computer using mathematical optimization technology.
\end{quote}
\vspace{5mm}
]

\section{はじめに}
複数の港で荷物を船に積み, 複数の港で降ろす運搬船を考える. 積荷の種類としては燃料, 原料など多岐に渡るが本研究では自動車を運搬する船を考える. 一般的に自動車運搬船は, 自動車を船の一定間隔で区切られたホールドと呼ばれるスペースにどの自動車を何台割り当てるかを考える席割作業を行い, その後席割作業で割り当てられた自動車に対して向きと場所考慮して一台ずつ船内の領域に配置するシミュレーション作業をする. 現状自動車を輸送する会社はこの作業を人手で行っているが少なくとも席割作業に3時間, シミュレーション作業に4時間かかることから, 多大な人件費と時間をかける必要があり直前の追加注文等に対応が出来ないという問題点がある\cite{SA}. 本研究ではこの問題を解決するために, 席割作業とシミュレーション作業それぞれについて数理最適化技術を用いることで, コンピューター計算により短時間で有効な席割とシミュレーションを出力することを目標とする. ただし本稿では研究の第一段階として, 席割作業とシミュレーション作業のうち席割作業の自動化に対して検証した結果を記す.

本研究で扱う席割作業の概要について述べる. 入力としてプリウスやハイエースなどの乗用車や, クレーン車やブルドーザーのような建機等の様々な種類の車を港Aから港Bまで輸送せよというような積載自動車の注文リストを受け取る. 注文リストを受け取ったプランナーと呼ばれる席割を作成する作業者は好ましい席割になるように, 注文の船内スペースへの割当を考える. 席割作成における前提条件について説明する. 例えば船内の特定の領域に積まれている自動車よりも奥に積まれている自動車が, ある港Cで降ろされて空きスペースが発生したとする. この場合次の港Dで積む自動車があればその自動車の積みやすさのために, 船内に積んだ自動車を奥側に詰めて手前の領域を確保するというような既に積まれた自動車の航海中の移動は原則しない. また自動車は全てのホールドに容易にアクセスすることは出来ない. 例えば本稿の実験で扱う自動車運搬船は12階構造の内5階のみに外部から自動車を入れるスロープがあり, 船内の奥側ホールドにアクセスしたい場合はそのホールドにアクセスする際に通過するホールドに十分な空きスペースがないとアクセスすることが出来ない.

本稿では第2章で席割問題に対する詳細な問題設定や, 本研究で扱う自動車輸送航海における専門用語について定義する.  第3章では注文に含まれる自動車一つ一つをどの領域に割り当てるかを最適化する数理モデルと, そのモデルに登場する本研究独自の制約や好ましい席割作成のための目的関数を説明する. 第3章で紹介する数理モデルは二種類あり, 一つ目は効果的な席割作成に対する考慮事項を全て目的関数で表したモデルである. 二つ目は一つ目で提案したモデルについて, 求解時間を抑えるために一つ目のモデルの目的関数の一部を制約化したものである. 入力として与えられる自動車の数は膨大になる場合があり, その場合これらのモデルでは入力情報が増加した場合に短い時間で求解できない可能性がある$[2]$. 第4章では入力情報が膨大な場合でも十分短い時間で求解ができるように, 入力で与えられた積載自動車の注文リストをグループ化したものに対して船内スペースへの割り当てを最適化するモデルを提案する. 第5章では提案した数理モデルと商用ソルバーを用いて, 実際の過去の航海データを基に求解実験を行う. またソルバーで得られた結果を席割作業を行なっている方々に評価してもらうことで, 提案した数理モデルの有効性について考察する.


\section{問題説明}
様々な国で複数の港を経由し, 自動車を輸送する運搬船を想定する. このとき乗用車100台を港Aから港B, トラック30台を港Cから港Dというような各注文を, 運搬船の階層毎に一定の広さで区切られたスペースへの割当を計画する.この作業を席割作業(stowage plan)\cite{SA}という. 本稿では席割作業の自動化の初期段階として, 単純な船の構造データを用いて実験する. また実際の席割業務では積載する自動車の種類が多様であるため, ショベルカーやブルードーザーなどの建機は自動車の高さや重さが乗用車とは大きく異なるので特定の領域にしか収容出来ないという問題がある. これに対して本稿では自動車を全て乗用車とし, 船内の任意領域においてどの自動車も積載が可能と言う条件で席割を作成する. この章では一般的な船への自動車積載における専門用語の定義, 席割作業における入力情報や出力情報の説明をする.



%%%%%%%%%%%%%%%%%%%%%%%%%%%%%%%%%%%%%%%%%%%%%%%%%%%%%%%%%%%% References
\begin{thebibliography}{99}
  \bibitem{SA} Kris Braekers,An Caris,and Gerrit K Janssens.Exact and meta-heuristic approach for a general heterogeneous dial-a-ride problem with multiple depots.Transportation Research Part B: Methodlological,Vol.67,pp.166--186,2014
\end{thebibliography}
\end{document}

% LocalWords:  ij Imahori Yagiura Ibaraki th Metaheuristics McGeoch Aarts lrr
% LocalWords:  Lenstra Chichester algorithmicx Fulkerson lrrcrr
